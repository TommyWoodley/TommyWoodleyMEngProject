\chapter{Introduction}
% 1-3 pages
% It's a good idea to try to write the introduction to your final report early on in the project. 
% However, you will find it hard, as you won’t yet have a complete story and you won’t know what your main contributions are going to be. 
% However, the exercise is useful as it will tell you what you don’t yet know and thus what questions your project should aim to answer. 
% For the interim report this section should be a short, succinct, summary of the project’s main objectives. 
% Some of this material may be re-usable in your final report, but the chances are that your final introduction will be quite different.  
% You are therefore advised to keep this part of the interim report short, focusing on the following questions: What is the problem, why is it interesting and what’s your main idea for solving it?  
% (DON'T use those three questions as subheadings however!  The answers should emerge from what you write.)
Hello \cite{greenwade93}
\section{Objectives}
% - Aims for using drones - "Forest Drones for Environmental Sensing" - why they are needed
% - Biggest Current Limitation - Limited Flight Time
% - Previous work focuses on perching of drones on tree branches
% - Allows data to be collected over a much longer period
% - Various ways have been proposed using different designs of drones
% - Previous work has attempted to automate the process of perching drones
% - Difficulties in simulation
% - Prompted the development of a single-shot approach
% - Aims to use recent advanacements in skills based reinforcement learning with demonstrations to learn to perform tasks.
\section{Challenges}
\section{Contributions}