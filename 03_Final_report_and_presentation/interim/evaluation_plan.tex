\chapter{Evaluation Plan}
% Evaluation plan (1-2 pages)
% Project evaluation is very important, so it's important to think now about how you plan to measure success. 
% For example, what functionality do you need to demonstrate?  
% What experiments to you need to undertake and what outcome(s) would constitute success?  
% What benchmarks should you use? 
% How has your project extended the state of the art? 
% How do you measure qualitative aspects, such as ease of use? 
% These are the sort of questions that your project evaluation should address; this section should outline your plan.

This project will focus on learning a UAV perching manoeuvre through reinforcement learning from demonstrations. 
The main objective is to develop an algorithm that generates successful trajectories and outperforms existing algorithms such as NACfD, DQLfD, SAC, and DQN in terms of learning efficiency, particularly on demonstration datasets. 
Additionally, we will evaluate the robustness of the algorithm against suboptimal or detrimental demonstration data. \\

\noindent
\textbf{Functional Objectives: }
The functionality that we would like to demonstrate is as follows:
\begin{itemize}
  \item \textbf{Trajectory Generation: } The algorithm should generate feasible and efficient trajectories for UAV manoeuvres.
  Feasibility here means that it is possible for a UAV to perform the provided manoeuvre and follow the trajectory.

  \item \textbf{Performance: } We will compare the learning performance and efficiency against standard algorithms in the provided background (NACfD, DQLfD, SAC, DQN).
  Here two of the comparitive algorithms will use the demonstration dataset and two will not.

  \item \textbf{Robustness to Poor Demonstration Data: } Assess the algorithm's performance with demonstration datasets containing varying percentages of suboptimal and detrimental demonstrations. (30\%, 50\%, 80\%).
  
\end{itemize}

\noindent
\textbf{Evaluation Criteria: }
The project will be evaluated based on the following criteria:

\begin{itemize}
  \item \textbf{Trajectory Success Rate: } The percentage of trajectories that successfully complete the manoeuvre in real-world experiments.
  \item \textbf{Learning Efficiency: } The time taken for the algorithm to converge to a successful policy.
  \item \textbf{Performance: } This can be measured through the average reward, trajectory length, thrust cost (energy usage), and smoothness (measuring the change in velocities).
  \item \textbf{Robustness: } These metrics should stay high and maintain performance even when trained with lower-quality demonstration data.
\end{itemize}

\noindent
\textbf{Benchmarks: }
The project will use the following benchmarks for evaluation:
\begin{itemize}
  \item \textbf{Standard Algorithm Performance: } Use the performance metrics from NACfD, DQLfD, SAC, and DQN as benchmarks for comparison.
  \item \textbf{Previous Perching Trajectory Generation: } Compare the generated trajectories with trajectories from previous work.
\end{itemize}

