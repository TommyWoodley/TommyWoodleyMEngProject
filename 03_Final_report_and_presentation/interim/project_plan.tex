\chapter{Project Plan}
% Project Plan (1-2 pages). 
% You should explain what needs to be done in order to complete the project and roughly what you expect the timetable to be. 
% Don’t forget to include the project write-up (the final report), as this is a major part of the exercise. 
% It’s important to identify key milestones and also fall-back positions, in case you run out of time. 
% You should also identify what extensions could be added if time permits.  
% The plan should be complete and should include those parts that you have already addressed (make it clear how far you have progressed at the time of writing).  
% This material will not appear in the final report.

\textbf{Aim: } The project aims to design and implement a framework capable of learning agile perching trajectories for unmanned aerial vehicles (UAVs) from a limited set of non-expert demonstrations. 
The framework's objective is to beyond the demonstrations provided to enhance energy efficiency while successfully completing the perching task.

\textbf{Progress Overview: } At the time of writing, the literature provided above has been reviewed and compared.
Additionally we have begun to look into potential simulators that could be used for modelling the tether dynamics.
We have also considered a collaboration with the Techical University of Munich as discussed in the potential extensions of the project.
From this date, we plan to begin work on the main body of the plan as detailed below.

\begin{itemize}
  \item \textbf{Demonstration Data Collection}
  
  Collect a diverse set of demonstration trajectories including expert, suboptimal, and incorrect demonstrations from real-world human-piloted flight.
  This will utilise a UAV equipped with a pendulum within the aerial robotics laboratory.
  A motion capture system can be used to capture the positions and velocities at the required timesteps of the drone and the tether.
  These will then be used to compare performance of the framework and understand how it responds to poor demonstration data.

  \item \textbf{Tether Dynamics Model}
  
  We aim to fully model the dynamics of the tether and the drone using a supervised learning model.
  This will significantly help in later work and simplify the learning required in later stages.
  This will effectively reduce the amount the agent needs to learn in the environment.

  \begin{enumerate}
    \item \textbf{Data Generation: } 
    A physics-based simulation engine will be used to generate a dataset that covers a range of scnerios of the tether interacting with the drone and the branch-like structure.
    This will use a python based physics engine, we are currently examining and comparing possible agents: Box2D, PyBullet, NVIDIA PhysX.
    Parameters will be systematically varried to build up a strong dataset.

    \item \textbf{Supervised Learning Model for Tether Dynamics: }
    Utilise a deep neural network to learn this model.
    This should be capable of prediciting the position of the tether's weight given the previous positions/velocities of the drone.
    This will need to be validated with new simualted data.
    Additionally this could be validated based on the data collected from the demonstration data set.
  \end{enumerate}

  \item \textbf{Learning from Demonstrations Model}
  
  Utilise a normalised actor-critic from demonstrations model, incorporating improvements in sampling and utilising demonstrations to speed up the learning process.
  A reward mechanism will need to be used to focus on milestones of the task like the number of wraps achieved and the disarmment of the drone.
  Utilised a prioritised replay of demonstrations mechanisms to help combat suboptimal and poor demonstration datasets.

  \item \textbf{Beyond Demonstraions: Improving Energy Efficiency}
  
  Integrate additional terms into the reward function and utilise reward shaping to encourage smoother more energy efficient manoeuvres.
  This can use a combination of energy usage such as the thrust cost calculation as well as penalising changes in velocities which will help generate a smoother curve.
  Experiment with different weightings to achieve this.

  \item \textbf{Evaluation Stage}
  
  Utilising the created model, generate a set of trajectories from a range of 3 dimensional starting positions.
  Conduct testing of these paths in a simulated platform like Gazebo.
  Utilise the policy in real world environment to validate the results achieved.
  This will aim to perform a full manuever including attaching onto a branch-like structure and then unravelling and departing.

\end{itemize}

\textbf{Possible Extensions}
\begin{itemize}
  \item \textbf{Collaboration with the TUM}
  
  Utilise the pod style drone they have designed and use this same model to design trajectories suitable.
\end{itemize}

\textbf{Timeline:}
The timetime for the project is shown in Tabl~\ref{table:project-timeline}.
Part time refers to during term periods where the project will be done alongside academic modules.
During this period I plan to spend 2 days per week on the project.
Full time refers to the period when I am able to dedicate my full time towards this project.
The timeline includes a 2 week break during which spring term exams will be done.

\begin{table}[h]
  \centering
  \begin{tabular}{|p{5cm}|p{3cm}|p{6cm}|}

  \hline
  Milestone & End Date & Time to Complete \\ \hline
  Demonstration Data Collection    & 
  10th March      & 
  6 Weeks (Part time) in parallel with Tether Dynamics Model.        \\ \hline

  Tether Dynamics Model  & 
  25th Feburary      & 
  4 Weeks (Part time) in parallel with Demonstration Data Collection.      \\ \hline

  Learning from Demonstrations Model    & 
  21st April      & 
  2 Weeks (Part time) + 4 Weeks (Full time) with break for exam period.        \\ \hline

  Beyond Demonstraions: Improving Energy Efficiency    & 
  5th April      & 
  2 Weeks (Full time)        \\ \hline

  Evaluation Stage    & 
  19th May      & 
  2 Weeks (Full time)        \\ \hline

  Report Writeup    & 
  17th June      & 
  4 Weeks (Full time)        \\ \hline
  \end{tabular}
  \caption{Project Timeline}
  \label{table:project-timeline}
\end{table}